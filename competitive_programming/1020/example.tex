% add --shell-escape to pdflatex arguments.
% add following key to have keyboard shortcuts
%{
%    "key": "shift+b",
%    "command": "commandId",
%    "when": "editorTextFocus"
%},
%{
%"key": "shift+B",
%"command": "editor.action.insertSnippet",
%"when": "editorLangId == latex && editorTextFocus",
%"args": {
%    "snippet": "\\textbf{${TM_SELECTED_TEXT}$0}"
%}
%}
% \inputminted[breaklines, breakafter=]{python}{./.py}


\documentclass[14pt]{extarticle}
\usepackage[left=0.5cm , right = 0.5cm, top=0.5cm]{geometry}
\usepackage{helvet}
\usepackage{parskip}
\usepackage{amsmath}
\usepackage{amssymb}
\usepackage{graphicx}
\usepackage[spanish]{babel}
\usepackage[dvipsnames]{xcolor}
\usepackage{tcolorbox} % above of the svg package
\usepackage{svg} 
\usepackage{hyperref}
\usepackage{minted}
\renewcommand{\sfdefault}{lmss}  % este activa la letra lmss
\renewcommand{\familydefault}{\sfdefault} % este activa la letra lmss
\sffamily % este activa la letra lmss
%\hyperlink{page.2}{Go to page 2}
%\newpage
%text on page 2
%\begin{figure}[H]
%  \centering
%  \includesvg[width=\textwidth]{plot.svg}
%\end{figure}

%\begin{minted}{csharp}
%    // single comment
%    \end{minted}

% f(n) = \begin{cases}
%    n/2  & n \text{ is even} \\
%    3n+1 & n \text{ is odd}
%  \end{cases}

%\begin{align}
%    \frac{d}{dx} \ln x &= \lim_{h\to 0} \frac{\ln(x+h) - \ln x}{h} \\
%    &= \ln e^{1/x} &&\text{How this follows is left as an exercise.}\\
%    &= \frac{1}{x} &&\text{Using the definition of ln as inverse function}
%   \end{align}


\begin{document}

\tableofcontents

\section{A}

Contar los $1's$  con fuerza bruta, la entrada lo permite, haciendo dos fors.

\section{B}

La idea es que debido a que eventualmente hay que insertar al número $x$, excepto si es $n$, posponer insertarlo lo más posible, porque una vez que este ya ese valor no puede ser candidato al mex del array.

\begin{enumerate}
    \item $x = 0$: poner $1, 2, \cdots , n-1, 0$.  
    \item $x = n$: poner $0, 1, \cdots, n-1$.
    \item En otro caso poner $0, 1, \cdots x-1, x+1, x+2, \cdots, n-1, x$. 
\end{enumerate}

\section{C}

Si todos los elementos del array $b$ son $-1$, entonces si se fija una suma $x$ válida se ha de satisfacer para todos los elementos del array que $0 \leq x - a[i] \leq k$, pues $x - a[i] = b[i]$. A partir de esto, observando los casos extremos, con esto quiero decir el máximo y el mínimo del array $A$, los valores válidos para la suma han de estar en el intervalo $[\max{A}, k + \min{A}]$.

Si un elemento del array $b$ no es $-1$, ya tengo la suma, por lo que queda verificar que las demás sumas que se tienen sean la misma, y que las posiciones que son $-1$, permitan ser reemplazadas por enteros en el intervalo $[0, k]$.

\section{D}

Insertar una flor es equivalente a ignorar a uno de los elementos del array $b$, por tanto, no se ignora ninguno, o se ignora uno. La respuesta es $0$ en la primera situación, en otro caso es el menor entre todos los que cumplen que si es ignorado, entonces se pueden recoger todas las flores.

Esto anterior provee de una solución por fuerza bruta, comprobar si se puede resolver el problema sin ignorar ningún $b[i]$, y después para cada uno comprobar si se puede resolver el problema si se ignora. Sería $O(mn)$, la idea es que se pueden usar sufijos y prefijos. 

Para cada posición $i$ del array $b$ computo la menor posición $t$ del array $a$ tal que hasta ella se pueden satisfacer las primeras $i+1$ condiciones, esto lo voy a guardar en un array $p$.

Análogamente, para cada posición $i$ del array $b$, del final al principio, calculo la mayor posición tal que desde el final hasta ella se pueden satisfacer las últimas $n - i$ condiciones, lo voy a guardar en un array $s$.

Para hallar estos dos arrays, usar dos punteros, uno para iterar por el array $a$, y otro para iterar por el array $b$.

En ambos casos si no es posible satisfacer la restricción pongo un $-1$, notar que una vez que se pone un $-1$ ya los restantes elementos tanto en los sufijos como en los prefijos han de ser $-1$.

$p[m-1] \neq -1$, me dice si es posible resolverlo sin ignorar ningún elemento.

Si quiero saber si ignorando el $i$ elemento del array $b$ puedo resolver el problema, es comprobar las siguientes condiciones:

\begin{enumerate}
    \item Si $i > 0$ que $p[i-1] \neq -1$.
    \item Si $i < m-1$ que $s[i+1] \neq -1$.
    \item Si $i > 0$ y $i < m-1$ comprobar que $p[i-1] < s[i+1]$.
\end{enumerate}

\section{E}

Para este problema, como el entero a buscar aparece exactamente una vez, en el subarray, hay una única forma en la que la búsqueda binaria va a llegar a él, en el caso en que éste esté en el intervalo $[l, r]$.

La búsqueda binaria en cada \textit{query} se puede simular porque es $O(\log n)$, la simulo, durante este proceso van a haber valores malposicionados, esto significa:

\begin{itemize}
    \item valores que deben ser $<$ entero
    \begin{itemize}
        \item están bien posicionados. $mb$
        \item están mal posicionados. $mm$
    \end{itemize}
    \item valores que deben ser $>$ entero
    \begin{itemize}
        \item están bien posicionados. $Mb$
        \item están mal posicionados. $Mm$
    \end{itemize}
\end{itemize}

Con esta información se puede calcular la respuesta, si $mm < Mm$, entonces se pueden intercambiar $2 \times mm$ índices.

\begin{itemize}
    \item valores que deben ser $<$ entero
    \begin{itemize}
        \item están bien posicionados. $mb$
        \item están mal posicionados. $0$
    \end{itemize}
    \item valores que deben ser $>$ entero
    \begin{itemize}
        \item están bien posicionados. $Mb + mm$
        \item están mal posicionados. $Mm - mm$
    \end{itemize}
\end{itemize}

Esto garantiza que todos los que deben ser $<$ que el entero ya están en posición, de los que deben ser mayor hay $Mm - mm$ que faltan por arreglar, del total de $(n-x)$ valores hay $Mb + mm$ que están bien posicionados. Por tanto hay solución si $(n-x) - (Mb+mm) \geq (Mm - mm)$, en cuyo caso la solución sería $2 \times Mm$.

En caso contrario la idea es análoga, la condición es $(x-1) - (mb + Mm) \geq (mm - Mm)$, la respuesta sería $2 \times mm$.

\end{document}

