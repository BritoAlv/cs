% add --shell-escape to pdflatex arguments.
% add following key to have keyboard shortcuts
%{
%    "key": "shift+b",
%    "command": "commandId",
%    "when": "editorTextFocus"
%},
%{
%"key": "shift+B",
%"command": "editor.action.insertSnippet",
%"when": "editorLangId == latex && editorTextFocus",
%"args": {
%    "snippet": "\\textbf{${TM_SELECTED_TEXT}$0}"
%}
%}

\documentclass[14pt]{extarticle}
\usepackage[left=2cm , right = 2cm, top=2cm]{geometry}
\usepackage{helvet}
\usepackage{parskip}
\usepackage{amsmath}
\usepackage{amssymb}
\usepackage{graphicx}
\usepackage[spanish]{babel}
\usepackage[dvipsnames]{xcolor}
\usepackage{tcolorbox} % above of the svg package
\usepackage{svg} 
\usepackage{hyperref}
\usepackage{minted}
\renewcommand{\sfdefault}{lmss}  % este activa la letra lmss
\renewcommand{\familydefault}{\sfdefault} % este activa la letra lmss
\sffamily % este activa la letra lmss
%\hyperlink{page.2}{Go to page 2}
%\newpage
%text on page 2
%\begin{figure}[htbp]
%  \centering
%  \includesvg{plot.svg}
%  \caption{svg image}
%\end{figure}

%\begin{minted}{csharp}
%    // single comment
%    \end{minted}

% f(n) = \begin{cases}
%    n/2  & n \text{ is even} \\
%    3n+1 & n \text{ is odd}
%  \end{cases}

%\begin{align}
%    \frac{d}{dx} \ln x &= \lim_{h\to 0} \frac{\ln(x+h) - \ln x}{h} \\
%    &= \ln e^{1/x} &&\text{How this follows is left as an exercise.}\\
%    &= \frac{1}{x} &&\text{Using the definition of ln as inverse function}
%   \end{align}


\begin{document}



@BritoAlv

\section{Constructivo}


\subsection{Implicas}

\begin{enumerate}
    \item 987C: Si $n$ es par, se pueden poner $x,x, x+1, x+1, ...$, cuando $n$ es impar, sucede que un numero obligatoriamente ha de usarse tres veces, por tanto habrán $3$ indices tales $x, y, z$ tales que sus diferencias sean cuadrados perfectos : $y - x = a^2$, $z - y = b^2, z - x = c^2$. Ahora si se hace brute-force para hallar las soluciones mas pequeñas a lo anterior. Se obtiene $(1, 17, 26), (2, 11, 27)$, $26$ es par, por lo que no importa, Hay solución al menos a partir de $27$, en efecto, para $27$ se puede construir manualmente, usando ambas triplas, y para $n > 27$, se le pone $x, x, x+1, x+1$ después de la posición $27$.
\end{enumerate}

\subsection{Con Corner Cases}

\begin{enumerate}
    \item 983B: Si $k$ esta en el principio o en el final, no siempre se puede. Se podría para $n = 1, n = 2, k = 1$, en otro caso, se puede degenerar a usar solamente $3$ divisiones, pero hay que tener en cuenta la paridad de $k$ para que los arrays en los que se divide tengan tamaño impar.
\end{enumerate}

\section{Formas de Pensar}

\subsection{Propiedades de el estado final o Invariantes}

\begin{enumerate}
    \item 983C : Lo invariante es que si $x$ es un valor del array final $x$ pertenece a el array inicial.  Se puede asumir que el array esta ordenado. Es solamente necesario que $a[0] + [1] > a[n-1]$, en el array final habrán dos valores que serán los menores de este, estos dos valores estaban en el array original y ademas han de estar en este en posiciones consecutivas. 
    
    Esto reduce el problema a fijar estos dos valores y ver que es necesario hacer para lograr que el array cumpla la condición : para cada $j$: se fija a $a[j-1], a[j]$ como los menores valores de el array final, entonces contar a cuantos de los restantes valores del array es necesario aplicarle la operación. \begin{minted}{c++}
        x <= a[j-1] ||  x >= a[j-1] + a[j]
    \end{minted}
    Escoger el $j$ que aporte la menor cantidad de operaciones.  
\end{enumerate}

\end{document}

