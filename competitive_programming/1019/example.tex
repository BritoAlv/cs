% add --shell-escape to pdflatex arguments.
% add following key to have keyboard shortcuts
%{
%    "key": "shift+b",
%    "command": "commandId",
%    "when": "editorTextFocus"
%},
%{
%"key": "shift+B",
%"command": "editor.action.insertSnippet",
%"when": "editorLangId == latex && editorTextFocus",
%"args": {
%    "snippet": "\\textbf{${TM_SELECTED_TEXT}$0}"
%}
%}
% \inputminted[breaklines, breakafter=]{python}{./.py}


\documentclass[14pt]{extarticle}
\usepackage[left=0.5cm , right = 0.5cm, top=0.5cm]{geometry}
\usepackage{helvet}
\usepackage{parskip}
\usepackage{amsmath}
\usepackage{amssymb}
\usepackage{graphicx}
\usepackage[spanish]{babel}
\usepackage[dvipsnames]{xcolor}
\usepackage{tcolorbox} % above of the svg package
\usepackage{svg} 
\usepackage{hyperref}
\usepackage{minted}
\renewcommand{\sfdefault}{lmss}  % este activa la letra lmss
\renewcommand{\familydefault}{\sfdefault} % este activa la letra lmss
\sffamily % este activa la letra lmss
%\hyperlink{page.2}{Go to page 2}
%\newpage
%text on page 2
%\begin{figure}[H]
%  \centering
%  \includesvg[width=\textwidth]{plot.svg}
%\end{figure}

%\begin{minted}{csharp}
%    // single comment
%    \end{minted}

% f(n) = \begin{cases}
%    n/2  & n \text{ is even} \\
%    3n+1 & n \text{ is odd}
%  \end{cases}

%\begin{align}
%    \frac{d}{dx} \ln x &= \lim_{h\to 0} \frac{\ln(x+h) - \ln x}{h} \\
%    &= \ln e^{1/x} &&\text{How this follows is left as an exercise.}\\
%    &= \frac{1}{x} &&\text{Using the definition of ln as inverse function}
%   \end{align}


\begin{document}

\tableofcontents

\section{A}

El mejor caso es una secuencia de tamaño $n$, que lo impide: que en la secuencia $y$ haya dos elementos iguales, como el producto par a par es común, esto va a implicar que los respectivos elementos en el array de la entrada sean iguales. Por tanto, en cualquier secuencia que se construya no se pueden seleccionar dos elementos iguales del array original, con esta afirmación se resuelve el ejercicio. Pues lo que queda es contar cuantos elementos distintos hay en el array.

Esto puede ser implementado en $O(n \log n)$ poniendo los elementos del array en un set, e imprimiendo la cantidad de elementos de este.

\section{B}

La observación es que la operación de revertir el string puede ahorrar a lo más $2$ operaciones.

La razón es que al cambiar los bordes del string como son a lo más $2$, ahí es donde posiblemente se ahorra, la parte interior del string, se revierte, pero esto no afecta a la cuenta de las operaciones.

Como la máquina empieza en $0$, no es simétrico, añadir un $0$ al principio del string resuelve este problema, y ahora la máquina empieza como mismo empieza el string.

Se puede ahorrar dos operaciones si y solo si en este string hay un substring de la forma $10 \cdots 10$ o $01 \cdots 01$, porque para ahorrar operaciones el comienzo y el final de string deben ser diferentes, lo que da lugar a dos casos posibles: 

$$10 \cdots 10 \rightarrow 11 \cdots 00$$
$$01\cdots01 \rightarrow 00\cdots11$$

Se puede ahorrar a lo más una operación si es de la forma: $01111000000$.

No se puede ahorrar nada si solamente hay de un número.

\section{C}

La mediana de un array es menor o igual que \( k \), si y solo si:

\begin{itemize}
    \item \( L \) es la longitud del array
    \item \( v \) es la cantidad de elementos del array \( \leq k \).
\end{itemize}

\[ 2v - L \geq 0 \]

Esto es a partir de la definición dada en el problema.

Entonces observar que los elementos del array, desde el punto de vista del problema, solamente importan si son o no \( \leq k \).

La condición del problema es equivalente a encontrar dos índices \( l, r \) tales que en al menos dos de los tres arrays la mediana sea \( \leq k \):

\begin{itemize}
    \item \( \rightarrow \) Si se cumple que en al menos dos de los tres arrays la mediana es \( \leq k \), entre las 3 medianas hay al menos dos \( \leq k \), y por tanto la mediana de esos tres valores va a ser \( \leq k \).
    \item \( \leftarrow \) Si la mediana de los tres arrays es \( \leq k \), entonces el menor número del array es \( \leq k \), de donde al menos dos arrays poseen mediana \( \leq k \).
\end{itemize}

Los candidatos para esos dos arrays con mediana \( \leq k \), denotados por \( X \), pueden ser:

\begin{itemize}
    \item XX-
    \item X-X
    \item -XX
\end{itemize}

Se puede saber si un prefijo del array, cumple la condición, pues comprobar la condición es \( O(1) \), análogamente con los sufijos.

Verificar el segundo caso se hace comprobando que el menor sufijo y el menor prefijo que cumplen la condición no se intersecan.

El tercer caso es lo mismo que el primer caso pero virando el array.

Sea \( f([a, b]) = 2(b-a+1) - (\text{cant valores} \leq k \text{ en el subarray } [a, b]) \)

Un subarray es válido si y solo si \( f([a, b]) \geq 0 \). Además notar que \( f([a, b]) = f([0, b] - [0, a-1]) \).

En el primer caso, la idea es iterar por los prefijos, cada vez que se encuentre uno válido, digamos que de \( [0, l] \), hace falta encontrar un índice \( r \) tal que el subarray \( [l+1, r] \) sea válido.

Esto es equivalente a preguntar si:

\[ f([0, r]) \geq f(0, l) \]

Porque \( f([0, r]) - f([0, l]) = f([l+1, r]) \).

Para encontrar este \( r \), la idea es mantener en un set todos los valores \( f([0, r]) \), y preguntar si en el set hay algún elemento que es \( \geq f(0, l) \). A medida que se itera hay que remover los que ya se revisaron, pues el \( r \) tiene que ser \( l < r < n-1 \).

\end{document}
